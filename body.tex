\section{Introduction} \label{sec:intro}

There is a perceived  risk that LSST  images may be snooped during transmission from the telescope. This concerns the images used for alert processing which are transmitted within a minute and provide a valuable resource for identifying moving objects including satellites. Though our processing will ignore satellites other parties may be interested in this information.

There have been several communications concerning encryption on the wire this done into in some detail in \citeds{DMTN-163}.


\subsection{Baseline }
The Rubin Observatory construction project has been built with academic level security in mind.
The
 Information classification policy \citeds{LPM-122} classifies data as  “User Protected”.
The DM Information Security plan \citeds{LDM-324} states the “majority of network traffic will not require confidentiality“

Since our astronomy data is research data with no intrinsic value no great efforts have been made to secure the
data nor the network it is traveling on.
The network has been designed, and now largely implemented, for high throughput not for high security.

The baseline is for encryption of controls but not data i.e. authentication is encrypted, data transmission is not.

{\bf It would be extremely useful if we agreed on the security rating of LSST data (or subsets of it)  as per NIST  \citedsp{nist800-60}}. This is also one of the first steps in \citedsp{FIPS200} called out in \citedsp{NIST.SP.800-171}.
Naively one would assume the security objective would be \emph{Availability}, the potential impact would be \emph{low} for confidentiality, availability and integrity. Hence  the Security Category (SC) would be \{low,low,low\} in NIST terms.

\section{Which data is sensitive ?} \label{sec:which}

In communications thus far and in the security summit held on 6$^{th}$ April 2020 all data has been considered.

We believe the vast majority of the 20TB of nightly images are not of a sensitive nature.
It would be useful to understand if this is so, especially if delaying this data some period of time is sufficient to desensitize it. If this is the case do we need to encrypt all data ?

If there are patches of the sky which are sensitive can they be avoided at certain times ? e.g. Geo synchronous orbit stripe.
Are there other rough orbits we should delay data from ?

Or is it really the entire northern sky?



\section{Network security}\label{sec:net}

\begin{figure}
\begin{center}
\includegraphics[width=1.0\textwidth]{NetworksFY22}
\caption{Rubin Observatory Network topology for FY22, the primary route is on dedicated lines, the back route is on shared equipment. Securing this beyond the level of the provider will be close to impossible.  \label{fig:net}}
\end{center}
% original https://docs.google.com/presentation/d/1wIN6Dj_rPn8TASBUkAm6_-Yh255Gz9VbwFi61t4LCFs/edit#slide=id.g55f7c0247e_0_2
\end{figure}

On the mountain and in the base facility LSST networks and computers are in rooms requiring ID card access and the compounds
have 24/7 security staff.
We could increase security but it would be costly and probably not very popular with staff i.e it would require raising the security level of the entire facility to something like FSL\footnote{\url{https://emilms.fema.gov/IS1172/groups/85.html}} 2 .

The transfers from Summit to Base and Base to NCSA go over private networks that are not part of the public Internet. We have our own dedicated fibers running from the mountain to the base \citedsp{LSE-78},
intercepting transfers would require physical access to the fibers - tapping those would probably disrupt our network at least temporarily.
For an added security we could also monitor fiber attenuation.
The network is also actively monitored with Bro\footnote{\url{https://www.corelight.com/about-bro/how-bro-works/}} system. Hence we would at least known something was up.
Though technically feasible by perhaps bribing or coercing staff somewhere this seems unlikely - a physical tap should also be noticed during fiber inspections.  The image as transferred over this line is also in an obscure format (though could be put together with some effort).


International transfers to IN2P3 are still being worked out at this time.
With SLAC as the USDF  ESNet could be used for this transfer as well making the security a more DOE centric issue.
 One may consider the open storage network \citep{osn} for this also though its not currently the baseline.


\section{Transmission security} \label{sec:trans}

It has been indicated, through DOE, that holding data for some time before releasing to the collaboration is an acceptable approach to securing the images - this can be ensured by limiting the LSST personnel who have access to the files during the night.
We would transmit images for alert processing in near realtime to NCSA. We have suggested a six hour delay - there has been no indication of how long the delay needs to be - is one hour sufficient to prevent detection of satellite maneuvers ?

Currently alert images from the base are to be transmitted using BBCP\footnote{BBCP is an alternative to Gridftp when transferring large amounts of data}. The control channel is encrypted but the data channel is open - this is considered secure in the scientific world \citep{bbcp}. Hence theoretically packets snooped on their way to SLAC  could be extracted and processed.


Our preference would be to do this in software using TLS based protocols such as HTTPS or FTPS to replace BBCP. This may cause some performance hit.
We could encrypt the data  channel. This would cost us in  compute but could be similar to compression and not push up the number of cores needed for transfer. We would also have some software modification and setup there are projects like WireGuard\footnote{\url{https://www.wireguard.com/}} intending to do this at the kernel level but they are not suitable for Rubin use yet.
There are also  examples of hardware solutions for this which would probably be affordable, INRIA in \cite{10.1007/978-3-642-45073-0_1} have build a FPGA based AES\cite{aes} implementation  which gives 100Gbit/s network rates with $\mu s$ additional latency. Table 3 of \cite{10.1007/978-3-642-45073-0_1} lists the hardware which looks modest enough - we work with INRIA already. There is no price given in the document, but considering the cost of components and a contract to put it together one \emph{might} consider it to be  under \$1M, but we would need to get a quote. This is probably more complex than we might want to get into.

We could encrypt the images themselves before transmission or even on the FPGA of the camera data acquisition (DAQ) system.  The DAQ is complex construction prone to delays so we would rather not jeopardize the project by introducing changes there. Encrypting the files outside would mean more CPU and I/O - it would add a delay in the alerts processing - it may take as much as 20\% of the alerts time budget (1 minute).  This would not be very welcome in the science community. This would require some effort on our side to add encryption to the processing chain but this should be order some hundreds \$K depending on how \emph{secure} it is required to be.

Should we need to keep BBCP we would have to turn to IPsec on the router and this would require router upgrades. See also \citeds{DMTN-163} for more encryption background.



\subsection{Keeping it in Chile}\label{sec:chile}
The original baseline was to do alert processing in Chile in the base facility. Some years ago support for computing in Chile was sub optimal but has improved significantly - there is space in the base facility to hold machines for this.
This would make the base data center the prime target of any data  attack so we may need to review security there.
However if we agree the base/summit are secure we could avoid on the wire snooping by doing alert processing in the base facility. We would also then have to hold images there for the appointed embargo time  before releasing them to the community. We would need to do some work on the cost impacts of this but they should not be insurmountable.

\section{Alerts}
There is also some fear of alerts for satellites. In general DM software masks streaks caused by satellites. There is a small chance for some geosynchronous orbiting satellites may pass as solar system objects and be alerted on.
There should be a small window of streak length were this may be in issue i.e. shorter it is a solar system object and alerted normally, longer it is a satellite, debris or meteorite  and not alerted on. The in between needs to be agreed as well as how to deal with it.

\seciton {Physical Security}

\subsection{Chile physical security} \label{sec:chileps}

The summit computer room is used exclusively by Rubin Observatory, however, 
the computer room at the base is shared with other NOIRLab programs. 

The administration of the base computer room belongs to NOIRLab

Rubin's computer rooms in Chile have the following security controls:
\begin{itemize}
\item Access to the computer room is controled by card readers. The system is administered by NOIRLab
\item Video surveillance of Rubin's computer racks. The surveillance platform is administered by Rubin
\item Computer racks are maintained always locked, both front and back door. 
\item Water and temperature sensors are located in several locations of the computer rooms.
\item Door locking sensors are located in racks with critical networking equipment
\end{itemize}


\section{High level summary}\label{sec:sum}
Though the reason behind the call for security and the level required remain totally unclear
NSF and DOE asked for a brief summary of possibilities.
One should also consider which data we are talking about see \secref{sec:which}.
Items here are not costed but an indication is given in terms of low (possibly within cost), moderate (some \$100Ks), high (>\$1M)
Any change should be properly costed.


\begin{longtable}{p{0.75\textwidth} p{0.25\textwidth}}\hline
\textbf{Security idea} & \textbf{Rough cost level}  \\\hline
 {\bf Delay some or all images.} Depending on how secure this needs to be and where it is done the cost scales.  & Low to Moderate.\\
{\bf Encrypt all images.} This would have to be done on the summit or in la Serena before hitting the long haul network. Possibly no new hardware needed but a change in software.  & Can be Low \\
{\bf Transmission Layer Encryption.} Network encryption would probably require new hardware. & Moderate \\
{\bf Physically securing the network.} This will be next to impossible and would probably require new network agreements. This however would be the only way to ensure no packet snooping. & Very high \\
{\bf Avoid certain time coordinates.} This would require changing the scheduler to provide more constraints on pointing. &  Moderate  \\\hline
\end{longtable}

