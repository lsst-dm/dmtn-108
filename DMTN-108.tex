

\documentclass[DM,authoryear,toc]{lsstdoc}
% lsstdoc documentation: https://lsst-texmf.lsst.io/lsstdoc.html

% Package imports go here.
\usepackage{collect}



% Local commands go here.
\newcounter{reccount} %counter for recomendations

\newenvironment{recenv}[1]
    { % empty more or less but we need it so we can refer to it.
	\refstepcounter{reccount}  %increment the counter for each recomendation
		    #1
    }

\definecollection{nrecs}

\newcommand{\rec}[4]{ %  handle,label (rec:label for references) ,title, text
    \begin{recenv}
    \vspace{5pt}
    \noindent \textbf{NR-\thereccount~#3}. \newline
    \newline \label{rec:#2}\noindent {#4}
    \vspace{5pt}
    \end{recenv}

	% collecting  for output :

        \begin{collect}{nrecs}{}{}
		\noindent \textbf{NR-\ref{rec:#2}~#3.}\\
		#4 (See page \pageref{rec:#2}.) \newline \vspace{5pt}
	\end{collect}
	\typeout{NR-\thereccount: #1: #2: #3: :END}
}


% To add a short-form title:
% \title[Short title]{Title}
\title{Security of Rubin Observatory data}

% Optional subtitle
% \setDocSubtitle{A subtitle}

\author{%
William O'Mullane
}

\setDocRef{DMTN-108}

\date{\today}

% Optional: name of the document's curator
% \setDocCurator{The Curator of this Document}

\setDocAbstract{%
This is a brief summary of on the transmission of images for alert processing . Specifically it discusses how secure the transfer is and if something more could be done in that area.
}

% Change history defined here.
% Order: oldest first.
% Fields: VERSION, DATE, DESCRIPTION, OWNER NAME.
% See LPM-51 for version number policy.
\setDocChangeRecord{%
  \addtohist{1}{YYYY-MM-DD}{Unreleased.}{William O'Mullane}
}

\begin{document}

% Create the title page.
% Table of contents is added automatically with the "toc" class option.

\mkshorttitle
%switch to \maketitle if you wan the title page and toc


% ADD CONTENT HERE ... a file per section can be good for editing
\section{Introduction} \label{sec:intro}


There is a perceived  risk that LSST  images may be snooped during transmission from the telescope. This concerns the images used for Alert processing which are transmitted within a minute and provide a valuable resource for identifying moving objects including satellites. Though our processing will ignore satellites other parties may be interested in this information.

\section{Network security}\label{sec:net}

On the mountain and in the base facility LSST networks and computers are in rooms requiring ID card access.Of course we could increase security but it would be costly to consider increasing the security level on the facilities.

The transfers from Summit to Base and Base to NCSA go over private networks that are not part of the public Internet. We have our own dedicated fibers running from the mountain to the base \citedsp{LSE-78}
intercepting transfers would require physical access to the fibers - tapping those would disrupt our network at leas temporarily i.e. we would at least known something was up.  Though technically feasible by perhaps bribing or coercing staff somewhere this seems unlikely - a physical tap should also be noticed during fiber inspections.


\section{Transmission security} \label{sec:trans}
As we understand it holding data for 12 to 24 hours before releasing to the collaboration is an acceptable approach to securing the images - this can be ensured by limiting the LSST personnel who have access to the files during the night.

Currently alert images from the base are to be transmitted using BBCP\footnote{BBCP is an alternative to Gridftp when transferring large amounts of data} - though the control channel is encrypted the data channel is open - this is considered quite secure in the scientific world \citep{bbcp}. Theoretically packets snooped on the their way to Illinois could be extracted and processed.


We could encrypt the entire channel using  128 bit encryption - this would be a significant cost in compute. Though there are plenty of examples of hardware solutions for this which would probably be affordable, INRIA in \cite{10.1007/978-3-642-45073-0_1} have build a FPGA based AES\cite{aes} implementation  which gives 100Gbit/s network rates with $\mu s$ additional latency. Table 3 of \cite{10.1007/978-3-642-45073-0_1} lists the hardware which looks modest enough - we work with INRIA already. There is not price given in the document but cost of components and a contract to put it together one \emph{might} consider to be  under \$1M.

We could encrypt the images themselves before transmission or even on the FPGA of the camera data acquisition (DAQ) system.  The DAQ is complex constriction prone to delays so we would rather not jeopardize the project by introducing changes there. Encrypting the files outside would mean more CPU and I/O - it would add a delay in the alerts processing - it may take as much as 20\% of the alerts time budget (1 minute).  This would not be very welcome in the science community.

\subsection{Keeping it in Chile}\label{sec:chile}
The original baseline was to do alert processing in Chile in the base facility. Some years ago support for computing in Chile was sub optimal but has improved significantly - there is space in the base facility to hold machines for this.  If we agree the base/summit are secure we could avoid on the wire snooping by doing alert processing in the base facility. We would also then have to hold images there for the appointed embargo time (12 or 24 hours) before releasing them to the community. We would need to do some work on the cost impacts of this but they should not be insurmountable.


\appendix
% Include all the relevant bib files.
% https://lsst-texmf.lsst.io/lsstdoc.html#bibliographies
\section{References} \label{sec:bib}
\renewcommand{\bibsection}{}
\bibliography{local,lsst,lsst-dm,refs_ads,refs,books}

%Make sure lsst-texmf/bin/generateAcronyms.py is in your path
\section{Acronyms used in this document}\label{sec:acronyms}
\addtocounter{table}{-1}
\begin{longtable}{|p{0.145\textwidth}|p{0.8\textwidth}|}\hline
\textbf{Acronym} & \textbf{Description}  \\\hline

AES & Advanced Encryption Service \\\hline
BBCP & BaBar Copy Program \\\hline
CPU & Central Processing Unit \\\hline
DAQ & Data Acquisition System \\\hline
DM & Data Management \\\hline
DMTN & DM Technical Note \\\hline
DOE & Department of Energy \\\hline
ESNET & Energy Sciences Network \\\hline
FPGA & Field-Programmable Gate Array \\\hline
FSL & Facility Security Level \\\hline
INRIA & French National Institute for computer science and applied mathematics \\\hline
LSE & LSST Systems Engineering (Document Handle) \\\hline
LSST & Large Synoptic Survey Telescope \\\hline
NCSA & National Center for Supercomputing Applications \\\hline
NIST & National Institute of Standards and Technology (USA) \\\hline
SC & Science Collaboration \\\hline
Summit & The site on the Cerro Pachón, Chile mountaintop where the LSST observatory, support facilities, and infrastructure will be built \\\hline
camera & An imaging device mounted at a telescope focal plane, composed of optics, a shutter, a set of filters, and one or more sensors arranged in a focal plane array \\\hline
\end{longtable}

\end{document}
